% Content included from main.tex under "Phase I Experimental Results"
% Section header defined in main.tex

This section presents Phase I experimental findings. The primary goal is validating measurement infrastructure and fail-closed governance, not demonstrating capability or convergence.

\subsection{CAL-EXP-3: Measurement Validation}

CAL-EXP-3 validates that $\Delta p$ (success rate proxy) is computable per cycle and that variance between experimental arms is measurable. The experiment compares:
\begin{itemize}[leftmargin=1.5em]
\item \textbf{Baseline (lr=0.0):} RFL disabled; policy static
\item \textbf{Treatment (lr=0.1):} RFL enabled; policy updated based on verifier outcomes
\end{itemize}

Both conditions exhibited oscillatory $\Delta p$ dynamics around the decision threshold. No convergence or uplift is claimed; the purpose is infrastructure validation. Full time-series plots are available in the evidence pack (ancillary material).

\subsection{Fail-Closed Governance}

Separate stress tests confirm that governance predicates trigger correctly under out-of-bounds conditions:
\begin{itemize}[leftmargin=1.5em]
\item \textbf{F5.2 (variance ratio):} Fires when inter-arm variance exceeds threshold
\item \textbf{F5.3 (windowed drift):} Fires when $\Delta p$ drift exceeds tolerance
\end{itemize}
When triggered, these predicates cap the claim level at L0 (no capability claim). This is the expected behavior for Phase I stress tests.

\subsection{Dual-Root Attestation}

The Mirror Auditor confirmed the integrity of the dual-root attestation mechanism. For the \experimentID{\valHtShortHash} snapshot, coverage was \valMirrorCoverage\%, with \valBlocksAudited{} blocks fully audited and verified.
