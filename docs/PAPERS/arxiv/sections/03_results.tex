\section{Results}
\label{sec:results}

This section presents the findings from our experimental campaign, focusing on the observable characteristics of the RFL mechanism and PL slicing on \projectname{}'s operational behavior.

\subsection{Phase I Baseline Observations}
Initial experiments explored the RFL mechanism's influence on system stability. As shown in Figure \ref{fig:rfl_abstention_reduction}, the RFL-enabled configurations exhibited an abstention rate of \textbf{\valAbstentionReduct} compared to the baseline.

\begin{figure}[htbp]
    \centering
    \includegraphics[width=0.85\linewidth]{figures/generated/cal_exp_3_success.png}
    \caption{CAL-EXP-3: Observed $\Delta p$ (success rate proxy) over cycles. Baseline (lr=0.0, blue) vs Treatment (lr=0.1, green). Phase I observation: both conditions exhibit oscillatory behavior around the threshold (red dashed line at 0.05). No convergence or uplift observed. Data source: \texttt{docs/evidence/cal\_exp\_3/}.}
    \label{fig:rfl_abstention_reduction}
\end{figure}

The \Ht{} dynamics, visualized in Figure \ref{fig:rfl_ht_convergence}, depict the trajectory of the system's internal state.

\begin{figure}[htbp]
    \centering
    \includegraphics[width=0.85\linewidth]{figures/generated/cal_exp_3_delta_p.png}
    \caption{CAL-EXP-3: $\Delta p$ dynamics over cycles (top) and windowed comparison (bottom). Baseline (lr=0.0, blue) vs Treatment (lr=0.1, green). \textbf{Phase I: no convergence observed.} The treatment condition shows lower variance but both conditions remain oscillatory. No smoothing applied. Data source: \texttt{docs/evidence/cal\_exp\_3/}.}
    \label{fig:rfl_ht_convergence}
\end{figure}

\subsection{PL Slicing Operational Characteristics}
Varying the PL slice configurations impacted computational efficiency and decision accuracy. The \textbf{Dynamic-Depth} PL slice configuration exhibited a transaction throughput of \valTpsAvg{} TPS with an error rate of $< \valErrorRateDynamic\%$. These observations highlight the characteristics of different PL designs.

\begin{table}[htbp]
    \centering
    \begin{tabular}{l|c|c|c}
        \textbf{Slice Configuration} & \textbf{Throughput (TPS)} & \textbf{Latency (ms)} & \textbf{Error Rate (\%)} \\
        \hline
        Static-Fixed & 10,200 & 4.5 & 0.015 \\
        Dynamic-Depth & \valTpsAvg & \valLatencyMean & \valErrorRateDynamic \\
        Heuristic-Lite & 14,100 & 2.1 & 0.120 \\
    \end{tabular}
    \caption{Performance metrics across varying Ponderation Layer (PL) slice configurations.}
    \label{tab:pl_slice_efficiency}
\end{table}

\subsection{Dual-Root Attestation}
The Mirror Auditor confirmed the integrity of the dual-root attestation mechanism. For the \experimentID{\valHtShortHash} snapshot, coverage was \valMirrorCoverage\%, with \valBlocksAudited{} blocks fully audited and verified.

\subsection{Observed Capability Metrics}
Across all tested configurations, the simple capability metrics provide an understanding of the system's operational envelope. The average transaction processing speed was recorded at \valTpsAvg{} TPS, with peak throughput reaching \valTpsPeak{} TPS under specific load conditions (burst mode).
