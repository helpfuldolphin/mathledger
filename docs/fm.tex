% MathLedger Field Manual
% Operations Guide for RFL Infrastructure and Phase II Experiments
%
% USAGE: % MathLedger Field Manual
% Operations Guide for RFL Infrastructure and Phase II Experiments
%
% USAGE: % MathLedger Field Manual
% Operations Guide for RFL Infrastructure and Phase II Experiments
%
% USAGE: % MathLedger Field Manual
% Operations Guide for RFL Infrastructure and Phase II Experiments
%
% USAGE: \input{docs/fm.tex} or compile standalone

\documentclass[11pt,a4paper]{article}
\usepackage[utf8]{inputenc}
\usepackage{booktabs}
\usepackage{hyperref}
\usepackage{xcolor}
\usepackage{framed}
\usepackage{enumitem}

\title{\textbf{MathLedger Field Manual}\\
\large Operations Guide for RFL Infrastructure}
\author{MathLedger Research Fleet}
\date{Version 1.0 --- \today}

\begin{document}

\maketitle

\tableofcontents

\newpage

%% ============================================================
%% PHASE II STATUS PARAGRAPH
%% ============================================================
\section{Phase Status Summary}

\begin{framed}
\textbf{Phase I} established the core RFL infrastructure and validated it
through a negative control experiment. The First Organism closed-loop test
demonstrated deterministic execution and dual-root attestation, but produced
\emph{no measurable uplift}---as expected for a symmetric (path-invariant)
environment.

\textbf{Phase II U2} defines a family of preregistered experiments on four
asymmetric uplift slices (goal-conditioned, sparse-reward, chain-depth,
multi-subgoal). These experiments are designed to test whether learned policies
can reduce epistemic risk relative to random baselines.

\textbf{As of Evidence Pack v1 and the current commit, no uplift claims are made.}
All uplift-related content in this manual and companion documents represents
\emph{design}, \emph{theory}, or \emph{preregistration}---not empirical results.
\end{framed}

%% ============================================================
%% WHERE TO LOOK BOX
%% ============================================================
\section{Where to Look}

\begin{framed}
\textbf{Key Resources for Phase II Operations}
\begin{description}[leftmargin=3.5cm, labelwidth=3cm]
  \item[U2 Runner] \texttt{experiments/run\_uplift\_u2.py} (to be implemented)
  \item[Phase II Curriculum] \texttt{config/curriculum\_uplift\_phase2.yaml}
  \item[Preregistration] \texttt{experiments/prereg/PREREG\_UPLIFT\_U2.yaml}
  \item[Manifest/Audit Tools] \texttt{scripts/verify\_config\_hashes.py}, \texttt{rfl\_gate.py}
  \item[Theory (U2 Slices)] \texttt{RFL\_UPLIFT\_THEORY.md} Sections 9--11
  \item[Uplift Plan] \texttt{docs/PHASE2\_RFL\_UPLIFT\_PLAN.md}
  \item[VSD Phase 2] \texttt{docs/VSD\_PHASE\_2.md} Section 0.5 (Uplift Evidence Gate)
\end{description}
\end{framed}

%% ============================================================
%% OPERATIONAL PROCEDURES
%% ============================================================
\section{Operational Procedures}

\subsection{Running Phase I Tests}

Phase I infrastructure can be exercised via:

\begin{verbatim}
# First Organism closed-loop test
pytest tests/integration/test_first_organism.py -v

# Determinism verification
pytest tests/integration/test_first_organism_determinism.py -v
\end{verbatim}

\subsection{Phase II Experiment Execution (Planned)}

Once Phase II tooling is implemented, the U2 experiment family will follow this protocol:

\begin{enumerate}
  \item Ensure preregistration document is finalized and hashed.
  \item Run baseline mode with random candidate ordering.
  \item Run RFL mode with learned policy ordering (same seed).
  \item Generate statistical summary and manifest.
  \item Archive results to Evidence Pack v2 structure.
\end{enumerate}

\textbf{Note:} As of this writing, the U2 runner (\texttt{run\_uplift\_u2.py})
is not yet implemented. Phase II experiments have not been executed.

\subsection{Audit and Verification}

\begin{itemize}
  \item \textbf{Config Hash Verification:} \texttt{python verify\_config\_hashes.py}
  \item \textbf{RFL Gate Check:} \texttt{python rfl\_gate.py --check}
  \item \textbf{Mirror Audit:} See \texttt{docs/VSD\_PHASE\_2.md} Section 0.5
\end{itemize}

%% ============================================================
%% SLICE REFERENCE
%% ============================================================
\section{Phase II Uplift Slices (Reference)}

The four Phase II slices are designed to create asymmetric environments where
policy-based ordering \emph{should} outperform random baselines. These are
\textbf{planned experiments, not yet executed}.

\begin{table}[h]
\centering
\begin{tabular}{lll}
\toprule
\textbf{Slice} & \textbf{Objective} & \textbf{Status} \\
\midrule
\texttt{slice\_uplift\_goal} & Goal-conditioned target & Designed \\
\texttt{slice\_uplift\_sparse} & Sparse reward environment & Designed \\
\texttt{slice\_uplift\_tree} & Chain depth derivations & Designed \\
\texttt{slice\_uplift\_dependency} & Multiple subgoals & Designed \\
\bottomrule
\end{tabular}
\caption{Phase II uplift slices (preregistered, not yet run).}
\end{table}

%% ============================================================
%% GLOSSARY
%% ============================================================
\section{Glossary}

\begin{description}
  \item[RFL] Reflexive Formal Learning --- the feedback loop that updates
    policy based on verifiable outcomes.
  \item[$H_t$] Composite attestation root at cycle $t$.
  \item[U2] Phase II uplift experiment family on asymmetric slices.
  \item[Negative Control] Phase I experiment on symmetric slice (no expected uplift).
  \item[Evidence Pack v1] Phase I artifacts with sealed attestations.
\end{description}

%% ============================================================
%% DOCUMENT HISTORY
%% ============================================================
\section*{Document History}

\begin{tabular}{lll}
\toprule
\textbf{Date} & \textbf{Version} & \textbf{Notes} \\
\midrule
\today & 1.0 & Initial field manual creation \\
\bottomrule
\end{tabular}

\end{document}
 or compile standalone

\documentclass[11pt,a4paper]{article}
\usepackage[utf8]{inputenc}
\usepackage{booktabs}
\usepackage{hyperref}
\usepackage{xcolor}
\usepackage{framed}
\usepackage{enumitem}

\title{\textbf{MathLedger Field Manual}\\
\large Operations Guide for RFL Infrastructure}
\author{MathLedger Research Fleet}
\date{Version 1.0 --- \today}

\begin{document}

\maketitle

\tableofcontents

\newpage

%% ============================================================
%% PHASE II STATUS PARAGRAPH
%% ============================================================
\section{Phase Status Summary}

\begin{framed}
\textbf{Phase I} established the core RFL infrastructure and validated it
through a negative control experiment. The First Organism closed-loop test
demonstrated deterministic execution and dual-root attestation, but produced
\emph{no measurable uplift}---as expected for a symmetric (path-invariant)
environment.

\textbf{Phase II U2} defines a family of preregistered experiments on four
asymmetric uplift slices (goal-conditioned, sparse-reward, chain-depth,
multi-subgoal). These experiments are designed to test whether learned policies
can reduce epistemic risk relative to random baselines.

\textbf{As of Evidence Pack v1 and the current commit, no uplift claims are made.}
All uplift-related content in this manual and companion documents represents
\emph{design}, \emph{theory}, or \emph{preregistration}---not empirical results.
\end{framed}

%% ============================================================
%% WHERE TO LOOK BOX
%% ============================================================
\section{Where to Look}

\begin{framed}
\textbf{Key Resources for Phase II Operations}
\begin{description}[leftmargin=3.5cm, labelwidth=3cm]
  \item[U2 Runner] \texttt{experiments/run\_uplift\_u2.py} (to be implemented)
  \item[Phase II Curriculum] \texttt{config/curriculum\_uplift\_phase2.yaml}
  \item[Preregistration] \texttt{experiments/prereg/PREREG\_UPLIFT\_U2.yaml}
  \item[Manifest/Audit Tools] \texttt{scripts/verify\_config\_hashes.py}, \texttt{rfl\_gate.py}
  \item[Theory (U2 Slices)] \texttt{RFL\_UPLIFT\_THEORY.md} Sections 9--11
  \item[Uplift Plan] \texttt{docs/PHASE2\_RFL\_UPLIFT\_PLAN.md}
  \item[VSD Phase 2] \texttt{docs/VSD\_PHASE\_2.md} Section 0.5 (Uplift Evidence Gate)
\end{description}
\end{framed}

%% ============================================================
%% OPERATIONAL PROCEDURES
%% ============================================================
\section{Operational Procedures}

\subsection{Running Phase I Tests}

Phase I infrastructure can be exercised via:

\begin{verbatim}
# First Organism closed-loop test
pytest tests/integration/test_first_organism.py -v

# Determinism verification
pytest tests/integration/test_first_organism_determinism.py -v
\end{verbatim}

\subsection{Phase II Experiment Execution (Planned)}

Once Phase II tooling is implemented, the U2 experiment family will follow this protocol:

\begin{enumerate}
  \item Ensure preregistration document is finalized and hashed.
  \item Run baseline mode with random candidate ordering.
  \item Run RFL mode with learned policy ordering (same seed).
  \item Generate statistical summary and manifest.
  \item Archive results to Evidence Pack v2 structure.
\end{enumerate}

\textbf{Note:} As of this writing, the U2 runner (\texttt{run\_uplift\_u2.py})
is not yet implemented. Phase II experiments have not been executed.

\subsection{Audit and Verification}

\begin{itemize}
  \item \textbf{Config Hash Verification:} \texttt{python verify\_config\_hashes.py}
  \item \textbf{RFL Gate Check:} \texttt{python rfl\_gate.py --check}
  \item \textbf{Mirror Audit:} See \texttt{docs/VSD\_PHASE\_2.md} Section 0.5
\end{itemize}

%% ============================================================
%% SLICE REFERENCE
%% ============================================================
\section{Phase II Uplift Slices (Reference)}

The four Phase II slices are designed to create asymmetric environments where
policy-based ordering \emph{should} outperform random baselines. These are
\textbf{planned experiments, not yet executed}.

\begin{table}[h]
\centering
\begin{tabular}{lll}
\toprule
\textbf{Slice} & \textbf{Objective} & \textbf{Status} \\
\midrule
\texttt{slice\_uplift\_goal} & Goal-conditioned target & Designed \\
\texttt{slice\_uplift\_sparse} & Sparse reward environment & Designed \\
\texttt{slice\_uplift\_tree} & Chain depth derivations & Designed \\
\texttt{slice\_uplift\_dependency} & Multiple subgoals & Designed \\
\bottomrule
\end{tabular}
\caption{Phase II uplift slices (preregistered, not yet run).}
\end{table}

%% ============================================================
%% GLOSSARY
%% ============================================================
\section{Glossary}

\begin{description}
  \item[RFL] Reflexive Formal Learning --- the feedback loop that updates
    policy based on verifiable outcomes.
  \item[$H_t$] Composite attestation root at cycle $t$.
  \item[U2] Phase II uplift experiment family on asymmetric slices.
  \item[Negative Control] Phase I experiment on symmetric slice (no expected uplift).
  \item[Evidence Pack v1] Phase I artifacts with sealed attestations.
\end{description}

%% ============================================================
%% DOCUMENT HISTORY
%% ============================================================
\section*{Document History}

\begin{tabular}{lll}
\toprule
\textbf{Date} & \textbf{Version} & \textbf{Notes} \\
\midrule
\today & 1.0 & Initial field manual creation \\
\bottomrule
\end{tabular}

\end{document}
 or compile standalone

\documentclass[11pt,a4paper]{article}
\usepackage[utf8]{inputenc}
\usepackage{booktabs}
\usepackage{hyperref}
\usepackage{xcolor}
\usepackage{framed}
\usepackage{enumitem}

\title{\textbf{MathLedger Field Manual}\\
\large Operations Guide for RFL Infrastructure}
\author{MathLedger Research Fleet}
\date{Version 1.0 --- \today}

\begin{document}

\maketitle

\tableofcontents

\newpage

%% ============================================================
%% PHASE II STATUS PARAGRAPH
%% ============================================================
\section{Phase Status Summary}

\begin{framed}
\textbf{Phase I} established the core RFL infrastructure and validated it
through a negative control experiment. The First Organism closed-loop test
demonstrated deterministic execution and dual-root attestation, but produced
\emph{no measurable uplift}---as expected for a symmetric (path-invariant)
environment.

\textbf{Phase II U2} defines a family of preregistered experiments on four
asymmetric uplift slices (goal-conditioned, sparse-reward, chain-depth,
multi-subgoal). These experiments are designed to test whether learned policies
can reduce epistemic risk relative to random baselines.

\textbf{As of Evidence Pack v1 and the current commit, no uplift claims are made.}
All uplift-related content in this manual and companion documents represents
\emph{design}, \emph{theory}, or \emph{preregistration}---not empirical results.
\end{framed}

%% ============================================================
%% WHERE TO LOOK BOX
%% ============================================================
\section{Where to Look}

\begin{framed}
\textbf{Key Resources for Phase II Operations}
\begin{description}[leftmargin=3.5cm, labelwidth=3cm]
  \item[U2 Runner] \texttt{experiments/run\_uplift\_u2.py} (to be implemented)
  \item[Phase II Curriculum] \texttt{config/curriculum\_uplift\_phase2.yaml}
  \item[Preregistration] \texttt{experiments/prereg/PREREG\_UPLIFT\_U2.yaml}
  \item[Manifest/Audit Tools] \texttt{scripts/verify\_config\_hashes.py}, \texttt{rfl\_gate.py}
  \item[Theory (U2 Slices)] \texttt{RFL\_UPLIFT\_THEORY.md} Sections 9--11
  \item[Uplift Plan] \texttt{docs/PHASE2\_RFL\_UPLIFT\_PLAN.md}
  \item[VSD Phase 2] \texttt{docs/VSD\_PHASE\_2.md} Section 0.5 (Uplift Evidence Gate)
\end{description}
\end{framed}

%% ============================================================
%% OPERATIONAL PROCEDURES
%% ============================================================
\section{Operational Procedures}

\subsection{Running Phase I Tests}

Phase I infrastructure can be exercised via:

\begin{verbatim}
# First Organism closed-loop test
pytest tests/integration/test_first_organism.py -v

# Determinism verification
pytest tests/integration/test_first_organism_determinism.py -v
\end{verbatim}

\subsection{Phase II Experiment Execution (Planned)}

Once Phase II tooling is implemented, the U2 experiment family will follow this protocol:

\begin{enumerate}
  \item Ensure preregistration document is finalized and hashed.
  \item Run baseline mode with random candidate ordering.
  \item Run RFL mode with learned policy ordering (same seed).
  \item Generate statistical summary and manifest.
  \item Archive results to Evidence Pack v2 structure.
\end{enumerate}

\textbf{Note:} As of this writing, the U2 runner (\texttt{run\_uplift\_u2.py})
is not yet implemented. Phase II experiments have not been executed.

\subsection{Audit and Verification}

\begin{itemize}
  \item \textbf{Config Hash Verification:} \texttt{python verify\_config\_hashes.py}
  \item \textbf{RFL Gate Check:} \texttt{python rfl\_gate.py --check}
  \item \textbf{Mirror Audit:} See \texttt{docs/VSD\_PHASE\_2.md} Section 0.5
\end{itemize}

%% ============================================================
%% SLICE REFERENCE
%% ============================================================
\section{Phase II Uplift Slices (Reference)}

The four Phase II slices are designed to create asymmetric environments where
policy-based ordering \emph{should} outperform random baselines. These are
\textbf{planned experiments, not yet executed}.

\begin{table}[h]
\centering
\begin{tabular}{lll}
\toprule
\textbf{Slice} & \textbf{Objective} & \textbf{Status} \\
\midrule
\texttt{slice\_uplift\_goal} & Goal-conditioned target & Designed \\
\texttt{slice\_uplift\_sparse} & Sparse reward environment & Designed \\
\texttt{slice\_uplift\_tree} & Chain depth derivations & Designed \\
\texttt{slice\_uplift\_dependency} & Multiple subgoals & Designed \\
\bottomrule
\end{tabular}
\caption{Phase II uplift slices (preregistered, not yet run).}
\end{table}

%% ============================================================
%% GLOSSARY
%% ============================================================
\section{Glossary}

\begin{description}
  \item[RFL] Reflexive Formal Learning --- the feedback loop that updates
    policy based on verifiable outcomes.
  \item[$H_t$] Composite attestation root at cycle $t$.
  \item[U2] Phase II uplift experiment family on asymmetric slices.
  \item[Negative Control] Phase I experiment on symmetric slice (no expected uplift).
  \item[Evidence Pack v1] Phase I artifacts with sealed attestations.
\end{description}

%% ============================================================
%% DOCUMENT HISTORY
%% ============================================================
\section*{Document History}

\begin{tabular}{lll}
\toprule
\textbf{Date} & \textbf{Version} & \textbf{Notes} \\
\midrule
\today & 1.0 & Initial field manual creation \\
\bottomrule
\end{tabular}

\end{document}
 or compile standalone

\documentclass[11pt,a4paper]{article}
\usepackage[utf8]{inputenc}
\usepackage{booktabs}
\usepackage{hyperref}
\usepackage{xcolor}
\usepackage{framed}
\usepackage{enumitem}

\title{\textbf{MathLedger Field Manual}\\
\large Operations Guide for RFL Infrastructure}
\author{MathLedger Research Fleet}
\date{Version 1.0 --- \today}

\begin{document}

\maketitle

\tableofcontents

\newpage

%% ============================================================
%% PHASE II STATUS PARAGRAPH
%% ============================================================
\section{Phase Status Summary}

\begin{framed}
\textbf{Phase I} established the core RFL infrastructure and validated it
through a negative control experiment. The First Organism closed-loop test
demonstrated deterministic execution and dual-root attestation, but produced
\emph{no measurable uplift}---as expected for a symmetric (path-invariant)
environment.

\textbf{Phase II U2} defines a family of preregistered experiments on four
asymmetric uplift slices (goal-conditioned, sparse-reward, chain-depth,
multi-subgoal). These experiments are designed to test whether learned policies
can reduce epistemic risk relative to random baselines.

\textbf{As of Evidence Pack v1 and the current commit, no uplift claims are made.}
All uplift-related content in this manual and companion documents represents
\emph{design}, \emph{theory}, or \emph{preregistration}---not empirical results.
\end{framed}

%% ============================================================
%% WHERE TO LOOK BOX
%% ============================================================
\section{Where to Look}

\begin{framed}
\textbf{Key Resources for Phase II Operations}
\begin{description}[leftmargin=3.5cm, labelwidth=3cm]
  \item[U2 Runner] \texttt{experiments/run\_uplift\_u2.py} (to be implemented)
  \item[Phase II Curriculum] \texttt{config/curriculum\_uplift\_phase2.yaml}
  \item[Preregistration] \texttt{experiments/prereg/PREREG\_UPLIFT\_U2.yaml}
  \item[Manifest/Audit Tools] \texttt{scripts/verify\_config\_hashes.py}, \texttt{rfl\_gate.py}
  \item[Theory (U2 Slices)] \texttt{RFL\_UPLIFT\_THEORY.md} Sections 9--11
  \item[Uplift Plan] \texttt{docs/PHASE2\_RFL\_UPLIFT\_PLAN.md}
  \item[VSD Phase 2] \texttt{docs/VSD\_PHASE\_2.md} Section 0.5 (Uplift Evidence Gate)
\end{description}
\end{framed}

%% ============================================================
%% OPERATIONAL PROCEDURES
%% ============================================================
\section{Operational Procedures}

\subsection{Running Phase I Tests}

Phase I infrastructure can be exercised via:

\begin{verbatim}
# First Organism closed-loop test
pytest tests/integration/test_first_organism.py -v

# Determinism verification
pytest tests/integration/test_first_organism_determinism.py -v
\end{verbatim}

\subsection{Phase II Experiment Execution (Planned)}

Once Phase II tooling is implemented, the U2 experiment family will follow this protocol:

\begin{enumerate}
  \item Ensure preregistration document is finalized and hashed.
  \item Run baseline mode with random candidate ordering.
  \item Run RFL mode with learned policy ordering (same seed).
  \item Generate statistical summary and manifest.
  \item Archive results to Evidence Pack v2 structure.
\end{enumerate}

\textbf{Note:} As of this writing, the U2 runner (\texttt{run\_uplift\_u2.py})
is not yet implemented. Phase II experiments have not been executed.

\subsection{Audit and Verification}

\begin{itemize}
  \item \textbf{Config Hash Verification:} \texttt{python verify\_config\_hashes.py}
  \item \textbf{RFL Gate Check:} \texttt{python rfl\_gate.py --check}
  \item \textbf{Mirror Audit:} See \texttt{docs/VSD\_PHASE\_2.md} Section 0.5
\end{itemize}

%% ============================================================
%% SLICE REFERENCE
%% ============================================================
\section{Phase II Uplift Slices (Reference)}

The four Phase II slices are designed to create asymmetric environments where
policy-based ordering \emph{should} outperform random baselines. These are
\textbf{planned experiments, not yet executed}.

\begin{table}[h]
\centering
\begin{tabular}{lll}
\toprule
\textbf{Slice} & \textbf{Objective} & \textbf{Status} \\
\midrule
\texttt{slice\_uplift\_goal} & Goal-conditioned target & Designed \\
\texttt{slice\_uplift\_sparse} & Sparse reward environment & Designed \\
\texttt{slice\_uplift\_tree} & Chain depth derivations & Designed \\
\texttt{slice\_uplift\_dependency} & Multiple subgoals & Designed \\
\bottomrule
\end{tabular}
\caption{Phase II uplift slices (preregistered, not yet run).}
\end{table}

%% ============================================================
%% GLOSSARY
%% ============================================================
\section{Glossary}

\begin{description}
  \item[RFL] Reflexive Formal Learning --- the feedback loop that updates
    policy based on verifiable outcomes.
  \item[$H_t$] Composite attestation root at cycle $t$.
  \item[U2] Phase II uplift experiment family on asymmetric slices.
  \item[Negative Control] Phase I experiment on symmetric slice (no expected uplift).
  \item[Evidence Pack v1] Phase I artifacts with sealed attestations.
\end{description}

%% ============================================================
%% DOCUMENT HISTORY
%% ============================================================
\section*{Document History}

\begin{tabular}{lll}
\toprule
\textbf{Date} & \textbf{Version} & \textbf{Notes} \\
\midrule
\today & 1.0 & Initial field manual creation \\
\bottomrule
\end{tabular}

\end{document}
