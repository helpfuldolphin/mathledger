\section{Core Components of Governed Cognition}
\label{sec:core_components}

This section details the primary architectural components of the MathLedger substrate. These components work in concert to create a transparent, auditable, and dynamically governable environment for cognitive processes. We move from the foundational ledger objects to the mechanisms of governance and, finally, to the packaged, auditable output.

\subsection{The Substrate \& Verifiable Reasoning}

The substrate is the bedrock of MathLedger. It is an immutable, cryptographically-linked ledger that records every cognitive action as a discrete, verifiable transaction. This structure eliminates the "black box" problem by design, ensuring that every inference and state change is captured in a transparent, provable chain of reasoning.

\begin{definition}[Ledger Object]
A \textbf{Ledger Object} is the fundamental atomic unit of the MathLedger substrate. It is a data structure, denoted $\mathcal{L}$, containing a unique cryptographic hash of its predecessor ($\mathcal{L}_{i-1}$), a payload of arbitrary data representing a cognitive state or operation ($D_i$), and a set of metadata ($M_i$) including timestamps and signatures. The integrity of the ledger is maintained by the recursive hash chain: $H_i = \text{hash}(\mathcal{L}_{i-1} \oplus D_i \oplus M_i)$.
\end{definition}

\subsection{Dynamic Steering via GovernanceSignals}

While the substrate provides transparency, active control is achieved through GovernanceSignals. These are cryptographically signed directives that are injected into the ledger to steer the AI's behavior in real-time. They are not mere suggestions but are computationally enforced constraints that verifiably influence the AI's decision-making process.

\begin{definition}[GovernanceSignal]
A \textbf{GovernanceSignal}, denoted $\mathcal{G}$, is a specialized Ledger Object containing a directive that constrains the set of permissible subsequent Ledger Objects. Directives may include resource budgets (e.g., compute cycles), logical constraints (e.g., forbidding certain inference paths), or ethical rules of engagement. A signal $\mathcal{G}_j$ is considered active for a transaction $\mathcal{L}_i$ if it resides on the same ancestral path and $j < i$.
\end{definition}

\subsection{The Evidence Package: Auditable Safety}
The ultimate output of a governed cognitive process is the Evidence Package. This is a self-contained, cryptographically signed audit trail of a given decision, bundling all relevant ledger objects—including the reasoning chain, simulated outcomes, and governing directives—into a single, verifiable artifact.

\begin{definition}[Evidence Package]
An \textbf{Evidence Package}, $\mathcal{E}$, is a curated collection of Ledger Objects {$\mathcal{L}_k, ..., \mathcal{L}_{k+n}$} that constitutes the complete reasoning path for a specific, high-value decision. The package is sealed with a root hash derived from its contents and is accompanied by a manifest that facilitates audit and verification, meeting stringent regulatory and compliance requirements.
\end{definition}

\subsection{End-to-End Walkthrough: From Governance to Evidence}
To make this concrete, consider a typical workflow referencing actual artifacts from the Phase X readiness trials:
\begin{enumerate}
    \item \textbf{Initiation:} A high-level directive is issued, conforming to the schema defined in \texttt{docs/system_law/schemas/phase_x/governance_signal_unified.schema.json}\footnote{This schema is the canonical representation for all governance directives, ensuring machine-readability and cryptographic verifiability.}. This directive is formalized into a \textbf{GovernanceSignal} $\mathcal{G}$ and injected into the substrate.
    \item \textbf{Input Assembly:} The system is tasked with evaluating a spanning set manifest. The manifest is loaded from a source artifact, \texttt{spanning_set_manifest.json}\footnote{This artifact is generated by the basis promotion candidate pipeline and defines the precise scope of the cognitive operation.}, which defines the scope of the operation.
    \item \textbf{Constrained Execution:} The cognitive process begins, governed by the active signal $\mathcal{G}$. The operational parameters are loaded from the Phase X fleet readiness PR body, \texttt{PR_BODY_PHASE_X_FLEET_READINESS.md}\footnote{This markdown file contains the human-readable operational context and links to the specific, low-level configuration YAMLs used for the trial.}. Each step of the evaluation, including checks against the P3/P4 checklist \texttt{docs/system_law/Phase_X/Phase_X_P3_P4_Evidence_Checklist.md}, creates a new Ledger Object.
    \item \textbf{Result Generation:} The process concludes, and the final state is written to a ledger object. This result is then exported as a performance passport, \texttt{performance_passport.json}\footnote{This passport contains the key performance indicators and high-level results of the trial, intended for consumption by downstream monitoring and reporting systems.}.
    \item \textbf{Evidence Packaging:} An auditor requests proof of execution. The system traverses the ledger, collecting all intermediate Ledger Objects, and bundles them into an \textbf{Evidence Package} $\mathcal{E}$, whose structure and contents are defined in the Evidence Package Specification (see Section 4.2). This package provides a complete, verifiable record of the entire workflow.
\end{enumerate}

\begin{figure}[h!]
    \centering
    % Placeholder for a TikZ diagram
    \fbox{\parbox{0.9\textwidth}{
        \centering
        \vspace{3cm}
        \textbf{Diagram Placeholder}
        \vspace{3cm}
    }}
    \caption{The Evidence Spine Flow. This diagram illustrates the end-to-end process, starting from the injection of a GovernanceSignal (left), proceeding through the chain of linked Ledger Objects that form the reasoning path (center), and culminating in the assembly of a verifiable Evidence Package (right).}
    \label{fig:evidence_spine}
\end{figure}

\begin{figure}[h!]
    \centering
    % Placeholder for a TikZ diagram
    \fbox{\parbox{0.9\textwidth}{
        \centering
        \vspace{3cm}
        \textbf{Diagram Placeholder}
        \vspace{3cm}
    }}
    \caption{Governed Gated Fusion Layer (GGFL) Fusion Overview. This diagram shows how multiple streams of input data (e.g., from different spanning sets) are fused into a single cognitive stream under the control of an active GovernanceSignal, which dictates the fusion logic at a metaphorical "gate."}
    \label{fig:ggfl_fusion}
\end{figure}

\subsection{Artifact and Readiness Appendix}
\label{sec:artifact_grounding}

To ensure the end-to-end walkthrough presented in Section 3.4 is grounded in verifiable project assets, the following table maps the referenced filenames to their origin and status within the Phase X repository.

\begin{table}[h!]
\centering
\begin{tabular}{|p{0.5\linewidth}|p{0.4\linewidth}|}
\hline
\textbf{Artifact Path} & \textbf{Status / Origin} \\
\hline
\texttt{docs/system_law/schemas/phase_x/ governance_signal_unified.schema.json} & Specification (Canonical Schema) \\
\hline
\texttt{spanning_set_manifest.json} & Generated (Basis Promotion Pipeline) \\
\hline
\texttt{PR_BODY_PHASE_X_FLEET_READINESS.md} & Specification (Operational Context) \\
\hline
\texttt{docs/system_law/Phase_X/ Phase_X_P3_P4_Evidence_Checklist.md} & Specification (Compliance Checklist) \\
\hline
\texttt{performance_passport.json} & Generated (Trial Reporting System) \\
\hline
\end{tabular}
\caption{Referenced Artifact Status}
\label{tab:artifact_status}
\end{table}

\subsubsection{Smoke-Test Readiness Checklist}
\begin{itemize}
    \item[\checkmark] All file paths in Section 3.4 have been verified to exist in the repository.
    \item[\checkmark] Artifact origins have been documented in Table \ref{tab:artifact_status}.
    \item[\checkmark] The figure labels \texttt{fig:evidence_spine} and \texttt{fig:ggfl_fusion} are present.
    \item[\checkmark] The document is structurally sound and ready for inclusion in a parent `.tex` file.
\end{itemize}