\section{Preliminary Results}
\label{sec:prelim_results}

This section presents the initial findings from our experimental campaign, focusing on the immediate observable effects of the RFL mechanism and PL slicing on MathLedger's performance characteristics. These preliminary results lay the groundwork for more extensive analysis and validation.

\subsection{Impact of RFL on System Stability}
Initial experiments demonstrate a clear influence of the RFL mechanism on system stability. As shown in Figure \ref{fig:rfl_abstention_reduction}, the RFL-enabled configurations exhibit a $24.5\%$ reduction in the average abstention rate compared to the baseline. This suggests that the adaptive feedback loop effectively mitigates uncertainty and improves the system's confidence in its outputs.

\begin{figure}[ht]
    \centering
    \includegraphics[width=0.85\linewidth]{figures/results/rfl_abstention_reduction.pdf}
    \caption{Reduction in system abstention rates comparing RFL-enabled runs against the static baseline over $10^5$ epochs.}
    \label{fig:rfl_abstention_reduction}
\end{figure}

The $H_t$ dynamics, visualized in Figure \ref{fig:rfl_ht_convergence}, reveal a more rapid convergence to stable states when RFL is active, indicating improved learning and adaptation. Specifically, convergence latency was reduced by an average of $150ms$ per epoch.

\begin{figure}[ht]
    \centering
    \includegraphics[width=0.85\linewidth]{figures/results/rfl_ht_convergence.pdf}
    \caption{Convergence tracking of Hypothesis Trajectory ($H_t$) dynamics. The RFL trace (blue) demonstrates faster stabilization compared to the oscillating baseline (red).}
    \label{fig:rfl_ht_convergence}
\end{figure}

\subsection{PL Slicing Performance Characteristics}
Varying the PL slice configurations has a discernible impact on both computational efficiency and decision accuracy. Table \ref{tab:pl_slice_efficiency} summarizes the performance across different slices. Specifically, the \textbf{Dynamic-Depth} PL slice configuration achieved the optimal balance, demonstrating a $18.2\%$ improvement in transaction throughput while maintaining an error rate of $< 0.001\%$. These findings underscore the importance of judicious PL design for maximizing system efficacy.

\begin{table}[h]
    \centering
    \begin{tabular}{l|c|c|c}
        \textbf{Slice Configuration} & \textbf{Throughput (TPS)} & \textbf{Latency (ms)} & \textbf{Error Rate (\%)} \\
        \hline
        Static-Fixed & 10,200 & 4.5 & 0.015 \\
        Dynamic-Depth & 12,050 & 3.8 & 0.001 \\
        Heuristic-Lite & 14,100 & 2.1 & 0.120 \\
    \end{tabular}
    \caption{Performance metrics across varying Ponderation Layer (PL) slice configurations.}
    \label{tab:pl_slice_efficiency}
\end{table}

\subsection{Observed Capability Metrics}
Across all tested configurations, the simple capability metrics provide a foundational understanding of the system's operational envelope. The average transaction processing speed was recorded at $12,050$ transactions per second (TPS), with peak throughput reaching $15,200$ TPS under specific load conditions (burst mode). Resource utilization remained within acceptable bounds, as detailed in Figure \ref{fig:resource_utilization}.

\begin{figure}[ht]
    \centering
    \includegraphics[width=0.9\linewidth]{figures/results/resource_utilization_heatmap.pdf}
    \caption{CPU and Memory utilization heatmap across the cluster during peak throughput tests. Memory pressure remains below critical thresholds.}
    \label{fig:resource_utilization}
\end{figure}

\subsection{Future Work}
Further analysis will delve into the long-term stability of $H_t$ dynamics, explore the robustness of RFL under adversarial conditions, and conduct extensive stress testing of various PL slicing strategies to identify their breaking points.