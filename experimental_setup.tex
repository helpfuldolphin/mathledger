\section{Experimental Setup}
\label{sec:exp_setup}

This section details the experimental methodology employed to evaluate the MathLedger system. Our primary focus is on assessing the performance and stability of the system under various configurations, particularly concerning the RFL (Reinforced Feedback Loop) mechanism and the PL (Ponderation Layer) slicing.

\subsection{Test Harness}
Experiments are conducted within a dedicated FO (Feedback-Optimized) cycle harness. This harness simulates realistic operating conditions, allowing for precise control over input parameters and comprehensive capture of output metrics. The harness is designed to ensure reproducibility and provide a consistent environment for comparative analysis.

\begin{figure}[h]
    \centering
    \includegraphics[width=0.9\linewidth]{figures/setup/fo_cycle_harness.pdf}
    \caption{Architectural overview of the FO cycle harness, illustrating the injection points for synthetic transaction loads and the telemetry collection pipeline.}
    \label{fig:harness_architecture}
\end{figure}

\subsection{Configurations Under Test}
We evaluate several key configurations:
\begin{enumerate}
    \item \textbf{RFL vs. Baseline:} We compare the performance of the MathLedger system with the RFL mechanism enabled against a baseline configuration where RFL is disabled or replaced with a static control mechanism. This comparison aims to quantify the impact of adaptive feedback on system stability and accuracy.
    \item \textbf{PL Slice Analysis:} The system's behavior is further investigated by varying the slicing of the Ponderation Layer. Different PL slice configurations are tested to understand their influence on computational efficiency, decision-making latency, and overall ledger integrity.
\end{enumerate}

\subsection{Measurement Metrics}
The following key metrics are recorded and analyzed during the experimental runs:
\begin{itemize}
    \item \textbf{Abstention Rate:} The frequency at which the system abstains from making a definitive judgment, indicating areas of uncertainty or insufficient confidence.
    \item \textbf{$H_t$ Dynamics:} We meticulously track the evolution and stability of the $H_t$ (Hypothesis Trajectory) dynamics over time, providing insights into the system's learning and adaptation processes.
    \item \textbf{Simple Capability Metrics:} A suite of fundamental performance indicators, including transaction processing speed, error rates, and resource utilization, are measured to establish a foundational understanding of system capabilities.
\end{itemize}